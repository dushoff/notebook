\section {Introduction}

In ecological studies, the large number of uncontrolled variables involved
often leads to levels of uncertainty that seem high compared to those in
less interesting fields, such as physics.  In some cases, this results in
ecologists looking bad, which is not acceptable \cite{EManifesto}.

For example, in pursuit of strict academic honesty, ecologists are
frequently reduced to saying things like ``the measured density of spruce
seedlings in the forest plots ($237/m^2$) and in the cornfield plots
($2/m^2$) were different at the 10\% level \cite{Eco1}.''  Meanwhile,
physicists are always saying things like ``the measured value of $2.2310403
\times 10^{-9}$ differs from the non-linearized quantum gauge approximation
$(2.2310166 \times 10^{-9})$ conclusively\footnote{We would print our
margin of error, but even if rendered in scientific notation, it would
occupy several journal pages {\em (note in original)}.} proving the
existence of the Hale-Gaborone effect \cite{Phy1}.  They do this purely to
annoy us \cite{PManifesto}.

The solution to this problem is the technique of multi-log transforms.
Multi-log transforms open up an exciting new world of precision to
ecologists.  

\section {The Fisher Transformation}

An example of a multi-log transformation is the Fisher transformation,
named to honor R. A. Fisher for his contributions to mathematical biology,
but primarily in order to attach a much-needed aura of legitimacy to this
approach.  The Fisher transformation is defined by:
$$
	F(x) = 1 + \log(\log(\log(\log(10^{100}x)))).
$$

The virtue of this transformation is that it takes values ranging from 1 to
$10^{12}$ and transforms them to a range from 0.4786 to 0.4936, greatly
improving precision.  For example, the sentence ``The difference between
the average wing lengths of crows (35 cm) and sparrows (12 cm) in our
sample was not statistically significant'' can now be written  ``The
difference between the transformed average wing lengths of crows (0.48068)
and sparrows (0.48007) in our sample was not statistically significant.''
The sentence, ``We feel the difference in average body mass between males
(250 kg) and females (95 kg) is biologically as well as statistically
significant" becomes ``We feel the difference in transformed average body
mass between males (0.48181) and females (0.48126) is biologically as well
as statistically significant."

The advantages are obvious.  Our level of precision has jumped from no
significant figures to three or four.

\section {Other transformations}

There are two problems with the Fisher transformation.  The first is that
it is not appropriate for data that may have zeroes.  The second, more
serious, problem, is that it could look suspicious if all of our numbers
started with 0.48.

For this reason, we need more multi-log transformations.  For the
transformations to achieve widespread use, it will also be necessary for
people to state in print that they are useful for particular purposes, so
that others can cite these assertions.  For example, I hereby suggest that
the Fisher transformation is appropriate for rationalizing morphological
measurements, such as lengths and masses.  I will also suggest that the
unnamed transformation below is an appropriate smoothing transformation for
`count' data.

The transformation 
$$ 
	D(x) = \log(1+\log(1+\log(1+\log(1+x)))))
$$
takes numbers ranging from 0 to $10^9$ into a range from 0 to 0.1143.  I
have been unable to think of a name for this transformation; maybe someone
else would like to suggest one in print.  Ideally, the transform should be
named after a pioneer in the field of multi-log transforms.

\section {Conclusions}

Multi-log transformations open up exciting new possibilities for precision
in ecology to rival that in annoying fields like physics.  For all we know,
physicists have used similar techniques in making their field look so
precise.  Why would they go around dragging `imaginary' numbers into all of
their papers if they weren't trying to hide something?  

We need people to develop more precision-enhancing transformations, and we
also need people who are willing to apply these transformations to their
data.  Techniques like this, combined with the current increasing trend in
obfuscatory language, promise to elevate ecology to the plane of the `hard'
sciences within the next few years.
